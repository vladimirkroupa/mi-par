\documentclass[12pt]{article}
\usepackage{epsf,epic,eepic,eepicemu}
%\documentstyle[epsf,epic,eepic,eepicemu]{article}
\usepackage[utf8]{inputenc}

\begin{document}
%\oddsidemargin=-5mm \evensidemargin=-5mm \marginparwidth=.08in
%\marginparsep=.01in \marginparpush=5pt \topmargin=-15mm
%\headheight=12pt \headsep=25pt \footheight=12pt \footskip=30pt
%\textheight=25cm \textwidth=17cm \columnsep=2mm \columnseprule=1pt
%\parindent=15pt\parskip=2pt

\begin{center}
\bf Semestrální projekt MI-PAR 2012/2013:\\[5mm]
    Paralelní algoritmus pro hledání minimální kostry grafu\\[5mm]
       Vladimír Kroupa\\
       Jan Stadler\\[2mm]
\end{center}

\section{Definice problému a popis sekvenčního algoritmu}

Nalezněte kostru grafu G (strom) s minimálním stupněm. Stupeň stromu je definován jako maximální stupeň ze všech uzlů stromu. Výstupem algoritmu je výpis kostry grafu G a hodnota stupně této kostry. Vstupem je  jednoduchý souvislý neorientovaný neohodnocený graf o $n$ uzlech a $m$ hranách, $n \in N, n \ge 5$, $k \in N, n \ge k \ge 3 $.

Řešení existuje vždy, vždy lze sestrojit kostru grafu. Sekvenční algoritmus je typu BB-DFS s hloubkou prohledávaného prostoru omezenou na $\vert$V$\vert$. Přípustný mezistav je definovaný částečnou kostrou. Přípustný koncový stav je vytvořená kostra. Algoritmus končí po prohledání celého prostoru či při dosažení dolní meze. Cena, kterou minimalizujeme, je stupeň kostry.

Těsná dolní mez je rovna 2. Horní mez je rovna nejvyššímu stupni v grafu G.

Pro usnadnění paralelizace řešení je použit explicitní zásobník, na který jsou ukládány jednotlivé stavy. Partikulární řešení se skládá z částečné kostry, reprezentované seznamem hran tvořících kostru a polem stupňů jednotlivých vrcholů, které částečná kostra obsahuje. Pole stupňů slouží k snadnému výpočtu maximálního stupně kostry (stačí pouze najít maximální prvek pole). Zároveň je použito při hledání hran, které lze do částečné kostry přidat. 

Použitý algoritmus je naivní, postupným přidáváním hran kostry prochází možné permutace hran, které vedou na kostru grafu. Použitím branch-and-bound strategie je ale významně zredukován počet procházených stavů.

Na zásobník se vkládají uzly grafu, které je možné přidat do existující částečné kostry. Mimo to se na zásobník vkládá symbol pro backtrack. Pokud je sejmut z vrcholu zásobníku tento symbol, je jasné, že jsme vyčerpali sourozenecké uzly stromu stavového prostoru a další prvek zásobníku bude rodičovský stav těchto sourozeneckých uzlů. Musíme tedy opravit partikulární řešení odstraněním poslední přidané hrany částečné kostry (což zahrnuje i aktualizaci pole stupňů vrcholů kostry).

Hlavní smyčka algoritmu probíhá následovně. Na počátku vložíme na zásobník hrany incidentní s vrcholem 0. Dokud není zásobník prázdný, snímáme z něj hrany. Po sejmutí prvku zkontrolujeme, zda se nejedná o backtrack symbol. Pokud ano, opravíme partikulární řešení. V opačném případě hranu přidáme do partikulárního řesení --- na konec seznamu hran částečné kostry. Otestujeme, zda částečná kostra již není kostrou grafu. To zjistíme snadno, počet hran kostry je roven počtu vrcholů grafu $-1$.

Pokud již máme řešení (kostru grafu), zjistíme její cenu (maximální vrchol kostry). Pokud je cena řešení rovná dolní mezi (2), ukončíme výpočet a vrátíme kostru. Pokud je cena vyšší než dolní mez, ale nižší, než cena nejlepšího doposud nalezeného řešení, aktualizujeme nejlepší řešení. Protože nyní se nacházíme v listu stromu stavového prostoru, opravíme partikulární řešení odstraněním poslední hrany.

Pokud řešení nemáme, expandujeme současný stav. Vložíme na zásobník symbol pro backtrack a dále vložíme všechny kandidátské hrany. Kandidátské hrany získáme tak, že nalezneme všechny hrany grafu, které mají jeden z vrcholů obsažen v aktuální částečné kostře a druhý vrchol v kostře doposud obsažen není. Každou z takto získaných hran zkusíme přidat do částečné kostry a vypočteme, jaký maximální stupeň by kostra měla po přidání této hrany. Ze seznamu těchto hran následně odstraníme ty, které by vedly na vyšší stupeň kostry, než je stupeň nejlepšího doposud nalezeného řešení.

Algoritmus terminuje buď nalezením řešení s cenou rovnou dolní mezi, nebo vyčerpáním zásobníku stavů. Po vyčerpání zásobníku musí nutně existovat alespoň jedno řešení.

Vstupem pro program je matice sousednosti, kde prvky jsou oddělené mezerami. Na prvním řádku před samotnou maticí je počet vrchovů grafu. Výstupem je kostra grafu a její vrchol.

\begin{figure}[ht]
\epsfysize=9cm \centerline{\epsfbox{fig/sequential.ps}} \caption{Doba běhu sekvenčního algoritmu pro graf $\vert V \vert = 70, \rho = 50$} \label{seq}
\end{figure}

\section{Popis paralelního algoritmu a jeho implementace v MPI}

Paralelní algoritmus je typu L-PBB-DFS-D. Každý procesor si udržuje informaci o lokálně nejlepším řešení a pokud řešení s cenou rovnou dolní mezi neexistuje, bude prohledávat celý stavový prostor a po distribuovaném ukončení výpočtu pomocí algoritmu ADUV bude globálně nejlepší řešení získáno paralelní redukcí lokálních řešení. Existuje-li řešení s cenou rovnou dolní mezi, procesor, který jej nalezne, ukončí výpočet vysláním zprávy typu jeden-všem.

Modifikace algoritmu pro paralelní zpracování je následující: Proces 0 načte matici sousednosti ze vstupního souboru. Poté rozešle ostatním procesům počet vrcholů grafu pomocí \texttt{MPI\_Bcast}. Nakonec rozešle pomocí \texttt{MPI\_Bcast} prvky matice. Proces 0 také na počátku výpočtu provede úvodní rozdělení práce ostatním procesům. Úvodní rozdělení funguje tak, že proces 0 expanduje stavy na zásobníku, dokud na zásobníku není tolik hran, kolik je procesů.

Hlavní změna oproti sekvenčnímu algoritmu je, že hlavní smyčka nemá jako podmínku ukončení neprázdný zásobník stavů, ale výsledek volání metody \texttt{shouldTerminate()}. Tato metoda se volá pouze jednou za \texttt{WORK\_STEPS} kroků, v ostatních případech algoritmus pokračuje v hledání. Tím je omezena režie na komunikaci mezi procesy. Parametr \texttt{WORK\_STEPS} je zadán jako argument při spuštění.

\subsection{Metoda \texttt{shouldTerminate()}}

Metoda \texttt{shouldTerminate()} nejdříve zkontroluje (\texttt{MPI\_Iprobe()}), zda proces nemá zprávu, požadující ukončení. Tuto zprávu vysílá proces 0 na konci distribuovaného ukončování výpočtu. Také ji vyslat jakýkoliv proces, pokud narazí na řešení s cenou rovnou dolní mezi (nejlepší možné řešení). 

Pokud povel k ukončení není přítomen, proces zjistí, zda má práci, o kterou se může podělit (nejméně 2 hrany na zásobníku stavů). Pokud nemá dostatek práce, zkontroluje a přijme (\texttt{MPI\_Iprobe(), MPI\_Recv()}) všechny požadavky na práci jemu adresované. Všem procesům od kterých přišly tyto požadavky zašle odmítnutí. Následně proces zkontroluje, zda má práci alespoň pro svůj vlastní běh (neprázdný zásobník). Pokud ano, metoda \texttt{shouldTerminate()} předá řízení zpět s návratovou hodnotou \texttt{false} a proces pokračuje v práci. V opačném případě (pokud proces má prázdný zásobník) proces vstoupí do režimu žádání o práci.
 
Pokud má proces práci, o kterou se může podělit, zkontroluje přítomnost požadavku na práci. Pokud mu je adresován požadavek, proces rozdělí svůj zásobník a zajistí přípravu rozdělené práce pro poslání delegováním na pomocnou třídu \texttt{Packer}. Proces posléze zkontroluje, zda nemá změnit svoji barvu (ta slouží pro distribuované ukončení). Zabalenou zprávu pošle žádajícímu procesu. Nakonec dojde návratu z metody \texttt{shouldTerminate()} s hodnotou \texttt{false} a proces pokračuje v práci.

Režim žádání o práci začíná tím, že proces zkontroluje, zda není jediným procesem ve skupině. Pokud ano, ukončí výpočet, žádná další práce už není. V opačném případě nejprve zkontroluje, zda mu nebyla adresován pokyn k ukončení výpočtu. Pokud ne, provede před samotnou žádostí o práci nejprve obsluhu distribuovaného ukončování.

Obsluha distribuovaného ukončování probíhá následovně: Proces nejprve zkontroluje, zda mu není adresován pešek. Pokud proces peška nedostal, je zároveň procesem s číslem 0 a zatím nezahájil proces distribuovaného ukončení vysláním bílého peška, vyšle bílého peška nyní. Pokud proces peška dostal, postup se opět liší podle toho, zda je o proces 0.

Proces s číslem 0 při získání peška zkontroluje, jestli je pešek bílý. Pokud ano, celé kolo distribuovaného ukončování proběhlo zdárně a proces 0 může vyslat všem procesům požadavek na ukončení. Pokud je pešek černý, jeden z procesů byl během posledního okruhu peška aktivní a ukončení nelze provést. V tomto případě se proces 0 pokusí o nové kolo ukončení vysláním bílého peška.

Ostatní procesy při získání peška přebarví peška na svoji barvu a pošlou peška dále. Veškeré posílání peška probíhá v kruhu. Jakékoliv posílání peška je tedy adresováno procesu s číslem o 1 vyšším, než je číslo odesílajícího procesu, modulo počet procesů.

Tím je obsluha pešků dokončena a proces vstupuje do stavu žádání o práci. Proces ve smyčce provádí následující: Protože smyčka obsahuje samé komunikační operace, je zde realizováno busy waiting, kdy se kontroluje stav inkrementujícího se čítače. Kontrolovaná hodnota odpovídá opět \texttt{WORK\_STEPS}. Proces nejprve rozešle všem procesům žádajícím o práci odmítnutí (sám žádnou práci v této fázi mít nemůže). Zkontroluje, zda mu nebyl adresován požadavek na ukončení. Pokud ano, ukončí se. Jinak obslouží pešky jako v minulém případě. Poté dojde k části řešící samotný požadavku na práci.

Proces nejprve zkontroluje, da už požadavek vyslal. Pokud ne, vyšle jej a pokračuje se dalším průchodem smyčkou. Pokud ano, proces ověří, zda neobdržel zprávu se sdílenou prací. Pokud ano, nastaví velikost bufferu na \texttt{BUFFER\_SIZE} a přijme zprávu. Pomocí třídy \texttt{Packer} zprávu rozbalí. Svoje partikulární řešení nahradí obdrženým partikulárním řešením a na svůj zásobník umístí přijaté hrany.

\subsection{Zpráva s rozdělenou prací}
Zpráva s rozdělenou prací má následující strukturu: Nejprve je hlavička, která obsahuje velikost zasílaného zásobníku a počet hran částečné kostry (partikulární řešení). Poté následují stupně jednotlivých vrcholů kostry (ty by bylo možné místo posílání také vypočítat lokálně), hrany kostry a nakonec stavy (hrany) rozděleného zásobníku.

\subsection{Dělení zásobníku, hledání dárce}
Dělení zásobníku probíhá ode dna. Proces prochází svůj zásobník ode dna, dokud nenarazí na hranu. Když narazí na hranu, projde zásobník od vrcholu k této hraně a počítá symboly pro backtrack, na které narazil. Toto číslo je potřeba k opravě partikulárního řešení, aby odpovídalo stavu, který bude ze zásobníku poslán (tolik hran je potřeba odstranit z konce částečné kostry).

Hledání dárce je realizováno algoritmem náhodné výzvy. Proces $P_{i}$ hledající práci si náhodně generuje číslo potenciálního dárce z množiny $\{1, .. ,p\}-\{i\}$. Generátor náhodných čísel je v každém procesu nainicializován aktuálním časem vynásobeným číslem procesu.

\section{Naměřené výsledky a vyhodnocení}

\begin{figure}[ht]
\epsfysize=9cm \centerline{\epsfbox{fig/parallel-35.ps}} \caption{Doba běhu paralelního algoritmu pro graf $\vert V \vert = 70, \rho = 35$} \label{seq}
\end{figure}

\begin{figure}[ht]
\epsfysize=9cm \centerline{\epsfbox{fig/parallel-50.ps}} \caption{Doba běhu sekvenčního algoritmu pro graf $\vert V \vert = 70, \rho = 50$} \label{seq}
\end{figure}

\begin{figure}[ht]
\epsfysize=9cm \centerline{\epsfbox{fig/parallel-60.ps}} \caption{Doba běhu sekvenčního algoritmu pro graf $\vert V \vert = 70, \rho = 60$} \label{seq}
\end{figure}

\begin{enumerate}
\item Zvolte tøi instance problému s takovou velikostí vstupních dat, pro které má
sekvenèní algoritmus èasovou složitost kolem 5, 10 a 15 minut. Pro
meøení èas potøebný na ètení dat z disku a uložení na disk
neuvažujte a zakomentujte ladící tisky, logy, zprávy a výstupy.
\item Mìøte paralelní èas pøi použití $i=2,\cdot,32$ procesorù na sítích Ethernet a InfiniBand.
\item Z namìøených dat sestavte grafy zrychlení $S(n,p)$. Zjistìte, zda a za jakych podmínek
došlo k superlineárnímu zrychlení a pokuste se je zdùvodnit.
\item Vyhodnoïte komunikaèní složitost dynamického vyvažování zátìže a posuïte
vhodnost vámi implementovaného algoritmu pro hledání dárce a dìlení
zásobníku pri øešení vašeho problému. Posuïte efektivnost a
škálovatelnost algoritmu. Popište nedostatky vaší implementace a
navrhnìte zlepšení.
\item Empiricky stanovte
granularitu vaší implementace, tj., stupeò paralelismu pro danou
velikost øešeného problému. Stanovte kritéria pro stanovení mezí, za
kterými již není uèinné rozkládat výpoèet na menší procesy, protože
by komunikaèní náklady prevážily urychlení paralelním výpoètem.

\end{enumerate}

\section{Závìr}

Celkové zhodnocení semestrální práce a zkušenosti získaných bìhem
semestru.

\section{Literatura}

\appendix

\section{Návod pro vkládání grafù a obrázkù do Latexu}

Nejjednodušší zpùsob vytvoøení obrázku je použít vektorový grafický
editor (napø. xfig nebo jfig), ze kterého lze exportovat buï
\begin{itemize}
\item postscript formáty (ps nebo eps formát) nebo
\item latex formáty (v poøadí prostý latex, latex s macry epic, eepic, eepicemu). Uvedené poøadí odpovídá rùstu
komplikovanosti obrázkù který formát podporuje (prostá latex macra
umožnují pouze jednoduché, epic makra nìco mezi, je tøeba
vyzkoušet).

\end{itemize}
Následující pøíklady platí pro všechny pøípady.

Obrázek v postscriptu, vycentrovaný a na celou šíøku stránky, s
popisem a èíslem. Všimnete si, jak øídit velikost obrazku.
%\begin{figure}[ht]
%\epsfysize=3cm \centerline{\epsfbox{VasObrazek.eps}} \caption{Popis
%vašeho obrazku} \label{labelvasehoobrazku}
%\end{figure}

%Obrázek pouze vložený mezi øádky textu, bez popisu a èíslování.\\
%\epsfxsize=1cm
%\rule{0pt}{0pt}\hfill\epsfbox{VasObrazek.eps}\hfill\rule{0pt}{0pt}

Latexovské obrázky maji pøípony *.latex, *.epic, *.eepic, a
*.eepicemu, respective.
%\begin{figure}[ht]
%\begin{center}
%\input VasObrazek.latex
%\end{center}
%\caption{Popis vašeho obrázku} \label{l1}
%\end{figure}
Vypuštením závorek {\tt figure} dostanete opìt pouze rámeèek v textu
bez èísla a popisu.

Takhle jednoduše mùžete poskládat obrázky vedle sebe.
%\begin{center}
%\setlength{\unitlength}{0.1mm}\input VasObrazek.epic
%\hglue 5mm
%\setlength{\unitlength}{0.15mm}\input VasObrazek.eepic
%\hglue 5mm
%\setlength{\unitlength}{0.2mm}\input VasObrazek.eepicemu
%\end{center}
Øídit velikost latexovskych obrázkù lze pøíkazem
\begin{verbatim}
\setlength{\unitlength}{0.1mm}
\end{verbatim}
které mìní mìøítko rastru obrázku, Tyto pøíkazy je ale souèasnì
nutné vyhodit ze souboru, který xfig vygeneroval.

Pro vytváøení grafu lze použít program gnuplot, který umí generovat
postscriptovy soubor, ktery vložíte do Latexu výše uvedeným
zpùsobem.

\end{document}
